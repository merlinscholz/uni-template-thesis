\chapter{Einleitung}
\label{chap:einleitung}

Dieses Dokument ist eine aktualisierte, modernisierte Version des \LaTeX-Templates, welches LS12 zur Verfügung stellt (\vgl \url{https://patrec.cs.tu-dortmund.de/cms/en/home/Resources/index.html}). Ich habe es für meine eigene Bachelorarbeit möglichst genau nachgestellt, da das originale Template seit einigen \LaTeX-Versionen nicht mehr kompilierbar ist.

\section{Zitierung}
\label{sec:einleitung/zitierung}
Eine Zitierung zum Test der Bibtex-Engine und Hyperrefs: \cite{test}.

\section{Tabellen}
\label{sec:einleitung/tabellen}
Test (und Beispiel) für das Einbinden von Tabellen, \vgl \autoref{tab:res_bsds500}.

\begin{table}[h!]
	\begin{tabularx}{\textwidth}{p{0.49\textwidth} >{\centering} p{0.25\textwidth} >{\centering\arraybackslash} p{0.25\textwidth}}
		\toprule
		\textbf{Segmentierungsmethode} & \textbf{Probabilistic Rand Index} & \textbf{Variation of Information} \\
		\midrule
		Mensch & $0,88$ & $1,17$ \\
		\midrule
		gPb-pwt-ucm & $0,83$ & $1,69$ \\
		Mean Shift & $0,79$ & $1,85$ \\
		Felz-Hutt & $0,80$ & $2,21$ \\
		Canny-owt-ucm & $0,79$ & $2,19$ \\
		NCuts & $0,78$ & $2,23$ \\
		Quad-Tree & $0,73$ & $2,46$ \\
		\textbf{Unüberwachtes tiefes Clustering} & $0,73$ & $2,88$ \\
		\bottomrule
	\end{tabularx}
	\caption{Metriken der Anwendung mehrerer Algorithmen auf den BSDS-500 Datensatz.}
	\label{tab:res_bsds500}
\end{table}