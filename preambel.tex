% In dieser Datei ist es eigentlich nicht nötig etwas überhalb der Markierung zu verändern
% Teilweise aus classicthesis (https://bitbucket.org/amiede/classicthesis/src/master/classicthesis.sty) übernommen,
% Anschließend durch den einsatz von KOMA modernisiert.

% Grundlegende Packages
\usepackage[utf8]{inputenc}
\usepackage[T1]{fontenc}
\usepackage{textcomp}
\usepackage[ngerman,german,english]{babel}


\usepackage[automark]{scrlayer-scrpage}


\KOMAoptions{twoside}
\KOMAoptions{titlepage}
\KOMAoptions{headinclude}
\KOMAoption{fontsize}{11pt}
\KOMAoption{paper}{a4}
\KOMAoptions{footinclude}
\KOMAoption{abstract}{false}
\KOMAoption{headings}{openright}
\KOMAoption{numbers}{noenddot}
\KOMAoptions{DIV=11, BCOR=10mm}

\setlength{\marginparwidth}{7em}%
\setlength{\marginparsep}{2em}%
\setcounter{secnumdepth}{2}

\DeclareTOCStyleEntries[raggedpagenumber=true,linefill={\hspace{1.5em}}]{tocline}{part,chapter,section,subsection}

% TODO Living headings vergleichen
% TODO TOC Style
% TODO Seitenzahlen
% TODO Dimensionen vergleichen
% TODO Subsubsection spacing
% TODO Paragraph font

% PDF Options
%\pdfcompresslevel=9
%\pdfadjustspacing=1 

% Schriftart
\usepackage{newpxtext} % TODO MATH FONT
\useosf
\usepackage[euler-digits]{eulervm}

% Farben
\usepackage[dvipsnames]{xcolor}
\definecolor{TUGreen}{rgb}{0.517,0.721,0.094}
\definecolor{BrightGray}{gray}{0.9}
\definecolor{DarkGray}{gray}{0.2}
\definecolor{white}{rgb}{1,1,1}
\definecolor{black}{rgb}{0,0,0}
\definecolor{red}{rgb}{1,0,0}
\definecolor{webgreen}{rgb}{0,.5,0}
\definecolor{webbrown}{rgb}{.6,0,0}

\usepackage{microtype}
\usepackage{textcase}
\usepackage{paralist}
\renewcommand{\chaptermark}[1]{\markleft{\textls[80]{\normalfont\scshape\MakeTextLowercase{#1}}}}
\renewcommand{\sectionmark}[1]{\markright{\textls[80]{\normalfont\scshape\MakeTextLowercase{#1}}}}

\setkomafont{disposition}{\rmfamily\mdseries}
\DeclareFixedFont{\chapterNumber}{U}{eur}{b}{n}{70}

\makeatletter

\renewcommand\chapterlinesformat[3]{%
\Ifstr{#1}{chapter}{%
	\vspace{-30pt}
	\mbox{
		\hspace{\linewidth}
		\smash{\vspace*{-3\baselineskip}\color{black!55}\chapterNumber#2}
	}
	\textls[80]{\MakeTextLowercase{#3}}
	\par
	\rule[.3\baselineskip]{\linewidth}{0.4pt}\par\nobreak
}{\@hangfrom{#2}{#3}}%
}

\renewcommand\sectionlinesformat[4]{
\@hangfrom{\hskip #2#3}%
\Ifstr{#1}{section}%
	{\textls[80]{\MakeLowercase{#4}}}%
	{#4}%
}%


\renewcommand\paragraphformat[1]{%
	{\textls[80]{\MakeLowercase{#1}}}%
}%

\makeatother

\RedeclareSectionCommand[
	font=\LARGE\scshape\mdseries
	]{chapter}
\RedeclareSectionCommand[
	font=\normalfont\scshape
	]{section}
\RedeclareSectionCommand[
	font=\normalfont\normalsize\itshape
	]{subsection}
\RedeclareSectionCommand[
	font=\normalfont\normalsize\itshape
	]{subsubsection}
\RedeclareSectionCommand[
	font=\normalfont\scshape
	]{paragraph}
\renewcommand{\sectionformat}{\thesection\hspace{1em}}
\renewcommand{\subsectionformat}{\thesubsection\hspace{1em}}
%\renewcommand{\subsubsectionformat}{\thesubsubsection asd\hspace{1em}}
\renewcommand{\descriptionlabel}[1]{\hspace*{\labelsep}{\textls[80]{\scshape\MakeTextLowercase{#1}}}}

% Hyperrefs und URLs
\usepackage{hyperref}
\usepackage{url}

% Misc Utility
\usepackage[cmex10]{amsmath}
\usepackage{amssymb}
\usepackage{mdwmath}
\usepackage{mdwtab}
\usepackage{eqparbox}
\usepackage{natbib}
\usepackage{xspace}

% Grafiken/Plots
\usepackage{flafter}
\usepackage{caption}
\usepackage{subcaption}
\usepackage{tikz}
\usetikzlibrary{arrows.meta}
\usepackage{wrapfig}
\usepackage{pgfplots}
\pgfplotsset{compat=1.15}
\usepgfplotslibrary{groupplots}

% Code Listings und Algorithms
\usepackage{minted}
\usepackage{algorithm}
\usepackage{algorithmic}

% Korrekte Darstellung der Umlaute
\usepackage{ae,aecompl}

% Anführungszeichen
\usepackage[babel,german=quotes]{csquotes}

% Setup und Finetuning
\newlength{\abcd} % Für ab..z Längenberechnung
\newcommand{\myfloatalign}{\centering} % Float alignment
\setlength{\extrarowheight}{3pt} % Höhere Tabellenzeilen

% Equation-Nummerierung nach <chapter>.<section>.<index>
\numberwithin{equation}{section} 
% Table-Nummerierung nach <chapter>.<section>.<index>
\numberwithin{table}{section} 
% Figure-Nummerierung nach <chapter>.<section>.<index>
\numberwithin{figure}{section} 

% Captions dem Original-Template anpassen
\captionsetup{format=hang,font=small}

% SI Units
\usepackage{siunitx}
\sisetup{
	locale = DE,
	per-mode = fraction 
}
\DeclareSIUnit\pixel{px}

% Tabellen
\usepackage{array}
\usepackage{booktabs}
\usepackage{tabularx}
\newcolumntype{L}[1]{>{\raggedright\let\newline\\\arraybackslash\hspace{0pt}}m{#1}}
\newcolumntype{C}[1]{>{\centering\let\newline\\\arraybackslash\hspace{0pt}}m{#1}}
\newcolumntype{R}[1]{>{\raggedleft\let\newline\\\arraybackslash\hspace{0pt}}m{#1}}


% Einbinden der Eidesstattlichen Versicherung
\usepackage[final]{pdfpages}

% Problematische Dateiendungen beim importieren
\usepackage{grffile}

% Autoref Algorithmus
\makeatletter
\renewcommand{\ALG@name}{Algorithmus}
\renewcommand{\listalgorithmname}{\ALG@name-Verzeichnis}
\makeatother

% Autoref Listing
\providecommand*{\listingautorefname}{Listing}


% Absatz Optimierungen
\clubpenalty = 10000
\widowpenalty = 10000

% Programming Font
\usepackage{courier}


\usepackage{todonotes}
\usepackage[nolist,nohyperlinks]{acronym}

% Anpassen von Plots
\pgfplotsset{ticklabel style={
		/pgf/number format/fixed,
		/pgf/number format/precision=4
	},scaled ticks=false}

\setminted{
	breaklines,
	fontsize=\footnotesize,
	frame=single
}
\setmintedinline{breaklines, breakafter=_}

% Abkürzungen
\newcommand{\tu}{TU Dortmund\@\xspace}
\newcommand{\vgl}{vgl.\@\xspace} 
\newcommand{\Vgl}{Vgl.\@\xspace}
\newcommand{\zB}{\mbox{z.\,B.}\xspace}
\newcommand{\bspw}{bspw.\@\xspace}
\newcommand{\idR}{\mbox{i.\,d.\,R.}\xspace}
\newcommand{\insb}{insb.\xspace}
\newcommand{\bzw}{bzw.\@\xspace}
\newcommand{\dahe}{\mbox{d.\,h.}\xspace}
\newcommand{\etc}{etc.\@\xspace}
\newcommand{\evtl}{evtl.\@\xspace}
\newcommand{\ggf}{ggf.\@\xspace}
\newcommand{\bzgl}{bzgl.\@\xspace}
\newcommand{\gdw}{gdw.\@\xspace}
\newcommand{\soben}{s.\nolinebreak[4]\hspace{0.125em}\nolinebreak[4]o.\@\xspace}
\newcommand{\sunten}{s.\nolinebreak[4]\hspace{0.125em}\nolinebreak[4]u.\@\xspace}
\newcommand{\iA}{i.\nolinebreak[4]\hspace{0.125em}\nolinebreak[4]A.\@\xspace}
\newcommand{\sa}{s.\nolinebreak[4]\hspace{0.125em}\nolinebreak[4]a.\@\xspace}
\newcommand{\ua}{u.\nolinebreak[4]\hspace{0.125em}\nolinebreak[4]a.\@\xspace}
\newcommand{\oa}{o.\nolinebreak[4]\hspace{0.125em}\nolinebreak[4]Ä.\@\xspace}
\newcommand{\og}{o.\nolinebreak[4]\hspace{0.125em}\nolinebreak[4]g.\@\xspace}
\newcommand{\oBdA}{o.\nolinebreak[4]\hspace{0.125em}\nolinebreak[4]B.\nolinebreak[4]\hspace{0.125em}d.\nolinebreak[4]\hspace{0.125em}A.\@\xspace}
\newcommand{\OBdA}{O.\nolinebreak[4]\hspace{0.125em}\nolinebreak[4]B.\nolinebreak[4]\hspace{0.125em}d.\nolinebreak[4]\hspace{0.125em}A.\@\xspace}
\newcommand{\etal}{\mbox{\textit{et.\,al.}}\xspace}


% Hyperrefs
\hypersetup{%
    colorlinks=true, linktocpage=true, pdfstartpage=3, pdfstartview=FitV,%
    breaklinks=true, pdfpagemode=UseNone, pageanchor=true, pdfpagemode=UseOutlines,%
    plainpages=false, bookmarksnumbered, bookmarksopen=true, bookmarksopenlevel=1,%
    hypertexnames=true, pdfhighlight=/O,%hyperfootnotes=true,%nesting=true,%frenchlinks,%
    urlcolor=RoyalBlue, linkcolor=RoyalBlue, citecolor=webgreen, %pagecolor=RoyalBlue,%
    pdftitle={\Titel},%
    pdfauthor={\textcopyright\ \Autor, \tu, \Faculty},%
    pdfsubject={},%
    pdfkeywords={},%
    pdfcreator={pdfLaTeX},%
    pdfproducer={LaTeX with hyperref}%
}
